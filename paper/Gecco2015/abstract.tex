The optimal mixing evolutionary algorithms (OMEAs) have recently drawn much attention for their robustness, small size of required population, and efficiency in terms of number of function evaluations (NFE).
In this paper, the performances and behaviors of OMEAs are studied by investigating the mechanism of optimal mixing (OM), the variation operator in OMEAs, under two scenarios---one-layer and two-layer masks.
For the case of one-layer masks, the required population size is derived from the viewpoint of initial supply, while the convergence time is derived by analyzing the progress of sub-solution growth.
NFE is then asymptotically bounded with rational probability by estimating the probability of performing evaluations.
For the case of two-layer masks, empirical results indicate that the required population size is proportional to both the degree of cross competition and the results from the one-layer-mask case.
%We also find that when applying disjoint masks corresponding to the problem structure,
%the supply model is a good estimator compared to other population-sizing models,
%the selection pressure imposed by OM makes the subproblem structure insignificant in terms of NFE,
%and the required population size for two-layer masks increases when the reverse-growth probability does.
The derived models also indicate that population sizing is decided by initial supply when disjoint masks are adopted,
that the high selection pressure imposed by OM makes the composition of sub-problems impact little on NFE,
and that the population size requirement for two-layer masks increases with the reverse-growth probability.
