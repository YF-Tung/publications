\documentclass{sig-alternate}
\usepackage{amssymb}
\usepackage{amsmath}
\usepackage{pgfplots}
%\usepackage{algorithm}
\usepackage[ruled,vlined]{algorithm2e}
\usepackage{algpseudocode}
%\usepackage{acmcopyright}

\renewcommand{\algorithmicrequire}{\textbf{Input:}}  % Use Input in the format of Algorithm
\renewcommand{\algorithmicensure}{\textbf{Output:}} % Use Output in the format of Algorithm
%\newdef{definition}{Definition}
\newtheorem{definition}{Definition}
\renewcommand{\thefootnote}{\fnsymbol{footnote}}

%\makeatletter
%\newcommand*{\myfnsymbolsingle}[1]{%
%    \ensuremath{%
%	\ifcase#1% 0
%%	    \or % 1
%%	    *%   
%	    \or % 2
%	    \dagger
%	    \or % 3  
%	    \ddagger
%	    \or % 4   
%	    \mathsection
%	    \or % 5
%	    \mathparagraph
%	    \else % >= 6
%	    \@ctrerr  
%	    \fi
%    }%   
%}   
%\makeatother
%
%\newcommand*{\myfnsymbol}[1]{%
%    \myfnsymbolsingle{\value{#1}}%
%}
%
%% remove upper boundary by multiplying the symbols if needed
%\usepackage{alphalph}
%\newalphalph{\myfnsymbolmult}[mult]{\myfnsymbolsingle}{}
%
%\renewcommand*{\thefootnote}{%
%    \myfnsymbolmult{\value{footnote}}%
%}

\begin{document}

\setcopyright{acmcopyright}
\conferenceinfo{GECCO '15,}{July 11 - 15, 2015, Madrid, Spain}
\isbn{978-1-4503-3472-3/15/07}\acmPrice{\$15.00}

\doi{http://dx.doi.org/10.1145/2739480.2754685}
%\setcopyright{acmcopyright}
%\conferenceinfo{GECCO '15,}{July 11 - 15, 2015, Madrid, Spain}
%\isbn{978-1-4503-3472-3/15/07}\acmPrice{\$15.00}
%
%\doi{http://dx.doi.org/10.1145/2739480.2754685}
%
%
%    \conferenceinfo{GECCO'15,} {July 11-15, 2015, Madrid, Spain.}
%    \CopyrightYear{2015}
%    \crdata{TBA}
    \clubpenalty=10000
    \widowpenalty = 10000

\title{Theoretical Perspective of Convergence Complexity of Evolutionary Algorithms Adopting Optimal Mixing}

\numberofauthors{2}
\author{
% 1st. author
\alignauthor
Yu-Fan Tung\\
       \affaddr{Taiwan Evolutionary Intelligence Laboratory}\\
       \affaddr{Department of Electrical Engineering}\\
       \affaddr{National Taiwan University}\\
       \email{r02921044@ntu.edu.tw}
% 2nd. author
\alignauthor
Tian-Li Yu\\
       \affaddr{Taiwan Evolutionary Intelligence Laboratory}\\
       \affaddr{Department of Electrical Engineering}\\
       \affaddr{National Taiwan University}\\
       \email{tianliyu@ntu.edu.tw}
}
%\date{14 November 2014}
\date{\today}

\maketitle
\begin{abstract}

The optimal mixing evolutionary algorithms (OMEAs) have recently drawn much attention for their robustness, small size of required population, and efficiency in terms of number of function evaluations (NFE).
In this paper, the performances and behaviors of convergence in OMEAs are studied by investigating the mechanism of optimal mixing (OM), the variation operator in OMEAs, under two scenarios---one-layer and two-layer masks.
For the case of one-layer masks, the required population size is derived from the viewpoint of initial supply, while the convergence time is derived by analyzing the progress of sub-solution growth.
NFE is then asymptotically bounded with rational probability by estimating the probability of performing evaluations.
For the case of two-layer masks, empirical results indicate that the required population size is proportional to both the degree of cross competition and the results from the one-layer-mask case.
%We also find that when applying disjoint masks corresponding to the problem structure,
%the supply model is a good estimator compared to other population-sizing models,
%the selection pressure imposed by OM makes the subproblem structure insignificant in terms of NFE,
%and the required population size for two-layer masks increases when the reverse-growth probability does.
The derived models also indicate that population sizing is decided by initial supply when disjoint masks are adopted,
that the high selection pressure imposed by OM makes the composition of sub-problems impact little on NFE,
and that the population size requirement for two-layer masks increases with the reverse-growth probability.

\end{abstract}

\category{F.2.m}{Analysis of Algorithms and Problem Complexity}{Miscellaneous}
%\category{G.1.6}{Numerical Analysis}{Optimization}[Global optimization]
\category{I.2.8}{Artificial Intelligence}{Problem Solving, Control Methods, and Search}[Heuristic methods]

% A category with the (minimum) three required fields
%\category{H.4}{Information Systems Applications}{Miscellaneous}
%A category including the fourth, optional field follows...
%\category{D.2.8}{Software Engineering}
%{Metrics}[complexity measures, performance measures]

\terms{Experimentation, Performance, Theory}

\keywords{
Optimal Mixing;
%Evolutionary Algorithms,
Convergence Complexity;
Population Sizing;
Convergence Time;
Number of Function Evaluations}

%% Chapter 1
\section{Introduction}
\label{sec:introduction}

The importance of convergence analysis of evolutionary algorithms cannot be overemphasized.
Without theoretical support, the empirical findings cannot be generalized,
and the development of new mechanism lacks direction.
However, due to the stochastic nature, the analyses of evolutionary algorithms (EAs)  are difficult,
and hence are either inaccurate and incomplete or come much later than the debut of the algorithm.
For example, the concept of the simple GA was proposed around the late 1960s.
Accurate facet-wise models for the population sizing and the convergence time were not proposed until the late 1990s~\cite{1992_Goldberg, 1993_Thierens, 1993_BGA, 1999_Gamblers, 2001_Supply}.
Similarly, the technique of model building in estimation of distribution algorithms (EDAs) has been developed since the late 1990s,
and complexity analysis of model building, the kernel of EDA, appeared in the field about ten years later~\cite{2002_Pelikan, 2007_Yu}.

Recently, the optimal mixing operator (OM), proposed by Thierens and Bosman,
has drawn much attention for its robustness and performance~\cite{2012_LN_OM_FI, 2013_Hier}.
Utilizing the intermediate results during variation, mixing with OM is considered noise-free decision making, which greatly reduces the required population size and boosts the performance.
The performance of OM significantly depends on the linkage sets~\cite{2012_LN_OM_FI,2012_Goldman}.
Recent researches involve using the CP index~\cite{2014_CP} and filtering/combining linkage hierarchies~\cite{2013_Filtering} to decide whether a given mask is promising during optimization.
However, those techniques have been developed with little theoretic support.

%Recently, a novel recombination operator, the optimal mixing (OM) operator, has drawn much attention for its robustness and performance~\cite{2012_LN_OM_FI, 2013_Hier}.
%Thierens and Bosman proposed OM as the variation operator adopted in the LTGA family~\cite{2011_OMEA}.
%Utilizing the intermediate results during variation, mixing with OM is considered noise-free decision making, which greatly reduces the required population size and boost the performance.
%Previous results also shows that performance significantly depends on the linkage sets utilized by OM.
%Recent researches involve using the CP index~\cite{2014_CP} and filtering/combining linkage hierarchies~\cite{2013_Filtering} to see whether a given mask is promising during optimization.
%Techniques have been developed with little theoretic support.

%Evolutionary algorithms (EAs) have been adopted as solutions to
%optimization problems for long, especially in black-box optimization settings.
%Although EAs are popular in real-world applications, 
%the performance of EAs is usually analyzed only based on empirical results.
%Theoretical models are often proposed much later.
%Take the example of the genetic algorithm (GA).
%GA was proposed in the 1970s,
%%but theoretical analysis is not complete until 2001 with a research about population sizing~\cite{2001_Supply}.
%but theoretical analysis was not complete until 2001~\cite{1992_Goldberg, 1993_Thierens, 1999_Gamblers, 2001_Supply}.
%Another example is the estimation of distribution algorithms (EDAs),
%which are proposed in the 1990s.
%The models for the Bayesian optimization algorithm (BOA) as well as other EDAs are proposed in 2002~\cite{2002_Pelikan},
%and the population-sizing model for general entropy-based EDAs is proposed in 2007~\cite{2007_Yu}.
%Deriving theoretical models for EAs may not be easy, since EAs are stochastic algorithms.
%Some analyses, like best-case and worst-case analyses, cannot directly be conducted on EAs.
%This may be the reason that the development of EA theories is considered slower than empirical studies.
%%Although empirical results are sufficient for applications,
%
%There are several reasons for deriving theoretical models.
%First, models derived by analyzing the mechanism of optimization are independent of problems,
%while empirical results depend on the benchmarks.
%Those results based on observation of known problems
%are not likely to fit well on unknown irregular problems, which are mainly the targets of EAs.
%Second, knowledge of theoretical complexities and optimization mechanisms facilitates the development of EAs.
%%In the framework of general EAs,
%Many operators have been proposed for selection and variation in EAs.
%Instead of simply compare them quantitatively, knowing the mechanisms help distinguish the strengths and weakness of them,
%which also helps develop innovative algorithms.
%For example, required population is larger to make correct decision~\cite{1992_Goldberg}
%than to ensure sufficient supply~\cite{2001_Supply}.
%The optimal mixing operator greatly reduces the demand of population size, 
%since the operator is noise-free and not bounded by decision-making models.
%%For example, some operators are suitable for solving problems with overlapping structures.
%Third, existing models can be foundations of more accurate models.
%Some facetwise models may seem not accurate at first,
%but combinations of them yields good results.
%For instance, the gambler's ruin model~\cite{1999_Gamblers}
%is a successive combination of supply models and decision models. 


%Although EAs are known for their robustness and potential,
%novice users may still be struggling when adopting them.
%First, determining the size of population is a critical issue.
%While smaller population often leads to low quality of solutions,
%using population with too large size may result in inefficient computation,
%gaining the cost of finding optimum.
%Second, the complexity of variants of EAs are not all well studied.
%For black-box optimization, no partial evaluation is allowed.
%Thus the cost of an algorithm is usually measured
%by number of function evaluations (NFE).
%Many innovative algorithms were proposed,
%but only empirical results of cost complexity are studied on most of them.
%Ignorance of complexity may make users hesitate about adopting these algorithms.

%Among all branches of EAs, the genetic algorithms (GA)~\cite{1975_Holland}
%have drawn a lot of attention.
%To estimate the required size of population, several model have been introduced.
%Goldberg~{\it et~al}.~\cite{2001_Supply} have proposed one from the viewpoint of supply.
%Thierens and Goldberg~\cite{1994_Thierens} have proposed an approximation of convergence time.
%Combination of these facetwise models leads to the model of required NFE.
%These models give researchers many insights on studies of many variants of EAs.

The goal of this paper is to analyze the convergence complexity of EAs adopting OM,
also called OMEAs, when various linkage models are adopted.
In particular, we focus on deriving analytical models for
the required population size, the convergence time, and the number of function evaluations (NFE).
%
%Many variants of OMEA are proposed,
%and one of them called GOMEA~\cite{2011_OMEA} most suits our study since we use the probabilistic model.
%
Among many OMEAs, the analyses in this paper suit best the scenario of GOMEA~\cite{2011_OMEA}
since our study heavily uses probabilistic models.

%Optimal mixing operators utilize intermediate function evaluations to ensure that fitness grows monotonically.
%For the promising performance and robustness of OMEAs~\cite{2012_LN_OM_FI, 2013_Hier},
%many variants have then been proposed, but the varation operator is the same, the optimal mixing operator.
%Many variants of OMEAs exist, but the adopted variation operator is the same.
%Therefore although we conduct experiments only on one variant called GOMEA,
%our results can be applied to all EAs adopting the operator.

%In this paper, we focus on a mixing operator in EAs.
%Successful design of mixing operator of EA involves two aspects.
%Partial solutions from different instances should be juxtaposed sufficiently,
%while variables in a partial solution should not be separated during mixing.
%Linkage learning is a process of finding which variables should be treated dependently during mixing.
%In 2010, Thierens proposed the linkage tree genetic algorithm (LTGA)~\cite{2010_LTGA},
%which detects the linkage relations in a hierarchical manner,
%and builds linkage trees utilizing the linkage information.
%%The step of building linkage trees makes the LTGA fall into the category of model-building GAs.
%In 2011, the ROMEA and the GOMEA, two variants of the LTGA, were proposed~\cite{2011_OMEA}.
%One of the difference between them and the canonical LTGA is a technique named optimal mixing (OM).
%Acting as the variation operator, OM significantly reduces the required population size and NFE.
%Some results~\cite{2012_LN_OM_FI, 2013_Hier} indicate these variants are robust and scalable on many problems.
%Specifically, the authors of LTGA have concluded that the GOMEA with linkage-tree learning is the new version of the LTGA,
%because the improvement in required population size, computing time, and number of function evaluations.
%In this paper, the performance and complexity of EAs adopting OM are analyzed.
%Specifically, we focus on gene-pool optimal mixing as the operator of OM,
%because it outperforms other variation operators.

For the rest of this paper, the background is first briefly introduced in Section~\ref{sec:relatedWorks}.
%and the algorithms and notations are introduced in Section~\ref{sec:algorithmsAndNotations}.
In the scenario of one-layer masks, complexity models are proposed in Section~\ref{sec:nonOverlappingMasks}.
The results are then extended to two-layer masks in Section~\ref{sec:twoLayerNonOverlappingMasks},
followed by conclusion.
%Finally, from the observations and discussions above,
%we conclude with a brief summary and suggestions for future works.
%Finally, the theoretical size of required population,
%convergence time, and number of function evaluations are derived and verified
%with experiments, and our conclusion follows.

%% Chapter 2
\section{Background}
\label{sec:relatedWorks}

In this section, background knowledge of this paper is provided.
Since the behavior of OMEAs depends on the linkage sets,
we first formally define the family of subsets and the associated notations.
Using the notations, the test problems used in this paper are then described.
%Following the definitions and notations,
%we introduce the framework of GOMEA,
%which is our research subject.
We then give a brief introduction to GOMEA, which is the research scenario of this paper.
%In addition, we introduced a previous population-sizing model,
%which is highly related to our results.
Finally, since the concept of initial supply is highly related to our work, it is also addressed.

%The general process of EA include several steps.
%For a given fitness function, the EA aims to find a solution with high fitness.
%First, candidate solutions are encoded as $\chi$-ary variable strings,
%which are called chromosomes.
%Next, a set of chromosomes are generated as the initial population.
%Then during the process of EA, the population evolves iteratively.
%In each iteration, some of the population are selected,
%juxtaposed, and used to produce new chromosomes,
%which replace the population then.
%These iterations are called generations.
%Finally, the population should converge to the desired result.
%To juxtapose the solutions,
%various types of variation operators can be adopted.
%The optimal mixing evolutionary algorithms (OMEA)~\cite{2011_OMEA}
%utilizes the OM operator in the variation step.
%
%Many researches have been done on the difficulty of EAs.
%Those results cannot be directly applied to our scenario,
%but similar methods and thoughts can be used.
%Some of them are briefly described below.

%Facetwise models have been proposed to estimate the complexity.
%From the models, we can understand the mechanisms of EAs,
%and design new algorithms by identifying and conquering
%the bottleneck of complexity.


%\section{Algorithms and Notations}
%\label{sec:algorithmsAndNotations}
\subsection{Family of Subsets}

In OMEAs, variables are mixed according to linkage sets
to ensure adequate mixing while preventing loss of promising partial solutions.
The family of subsets~\cite{2010_LTGA}, denoted by $\mathcal F$, is a general model of linkage sets.
Before defining the FOS we introduce the necessary notations.
%In the process of EA, certain number of candidate solutions are kept as the population.
%The candidates are called chromosomes, and each of them contains $\ell$ variables.
We denote the length of chromosome by $\ell$, and the population size by $n$.
%Those variables are also called genes.
Variables in one chromosome, also called genes, can be expressed in a vector $\vec{x} = \langle x_1, \dots, x_\ell\rangle$.
%Genes can be expressed in a vector $\vec{x}$.
%In this paper, the operator $|\ |$ over a vector denotes the number of elements in the vector.
The notation $|{\vec x}|$ denotes the number of elements in $\vec x$.
%Therefore we have $|\vec{x}| = \ell$,
The natural number set is denoted by $\mathbb N$.
Using these notations the FOS is defined as follows.
%The indexes of them forms a set, which is $S=\{1,2,\dots,\ell$\}.
%Subsets of $S$ are called masks.

\begin{definition}
For any integer $\ell\in{\mathbb N}$, $S_\ell$ is an {\bf index set} which consists of all integers from $1$ to $\ell$.
Subsets of $S_\ell$ are called {\bf masks}.
%A subset of the index set is a {\bf mask}.
\begin{equation*}
%S_\ell = \{i\in{\mathbb N}|1\leq i\leq \ell\}.
S_\ell = \{ 1, 2, \dots, \ell\}.
\end{equation*}
\end{definition}

\begin{definition}
For an index set $S_\ell$, $\mathcal F$ is a {\bf{family of subsets (FOS)}} with the following two properties.
\begin{enumerate}

\item $\mathcal F$ is an ordered set consisting of masks.
\begin{equation*}
\begin{split}
{\mathcal F} = \langle {\mathbf F}^1, &{\mathbf F}^2, \dots, {\mathbf F}^{|{\mathcal F}|}\rangle,\\
\text{where } {\mathbf F}^i &\subseteq S_\ell, 1\leq i \leq |{\mathcal F}|.
\end{split}
\end{equation*}

%\item Every element in $S_\ell$ must be contained in at least one of the masks in $\mathcal F$.
\item The union of all masks in $\mathcal F$ is the index set.
\end{enumerate}
\begin{equation*}
%\forall i \in S_\ell, \exists {\mathbf F}^j \in {\mathcal F}: i \in {\mathbf F}^j.
\bigcup\nolimits_{{\mathbf F}^i \in {\mathcal F}} {\mathbf F}^i = S_\ell.
\end{equation*}
\end{definition}

%For example, take $\ell = 3$.
Take $\ell = 3$ as an example.
The index set $S_3 = \{1,2,3\}$.
Sets such as $\{1\}$, $\{2,3\}$, and $\{1,2,3\}$ are all masks of $S_3$.
$\langle\{1,2\},\{2,3\}\rangle$, $\langle\{2,3\},\{1,2\}\rangle$, and $\langle\{1,2\},\{2,3\},\{1,2\}\rangle$ are three different FOSs of $S_3$.
$\langle \{1,2\},\{2\}\rangle $ is not a valid FOS of $S_3$, since no mask contains $3$.
Note that the masks in FOS are in order, and there may be duplicated masks.

%The FOS is an ordered set consists of some masks, so it is an ordered subset of the superset of $S$.
%In addition, every index should be contained in at least one mask in any legal FOS.
%Thus an FOS can be written by ${\mathcal F} = <{\mathbf F}^1, {\mathbf F}^2, \dots, {\mathbf F}^{|{\mathcal F}|}>$,
%${\mathbf F}^i \subseteq S, i \in \{1,2,\dots,|{\mathcal F}|\}$,
%with the constraint that $\forall i \in S, \exists j$ such that $i \in {\mathbf F}^j$.

%A mask is a set consisting of certain variables with regarding to the indexes.
%During optimal mixing, %the information of variables is exchanged with regarding to the variable indexes.
%information on the variables indicated by the masks is exchanged.
%These operations can be viewed as variable-wise mask operations.
%Therefore, the sets containing the variable indexes are called masks.
%For a chromosome $C$, the variables with indexes contained in a mask $M$ are denoted by $C_M$.

During OM, variables are mixed
according to the elements in FOS.
The operation can be viewed as a variable-wise mask operation,
so elements in FOS are called masks.
For a chromosome $C$, the variables selected according to a mask $M$ are denoted by $C_M$.

Some special FOSs are further defined as follows.
%We further denote some specific FOSs by the symbols defined below.
%These notations will be frequently used in the following context.

\begin{definition}
For an index set $S_\ell$ and an integer $k$ dividing $\ell$,
a {\bf{homogeneous FOS}} ${\mathcal F}_k$ is defined as
\begin{equation*}
{\mathcal F}_k = \langle{\mathbf F}^1, {\mathbf F}^2, \dots, {\mathbf F}^{\ell / k}\rangle,
\end{equation*}
\begin{equation*}
%\text{where } {\mathbf F}^i = \{i\cdot k-k+1, i\cdot k-k+2, \dots, i\cdot k\}
%\text{where } {\mathbf F}^i = \{\pi_{i\cdot k-k+1}, \pi_{i\cdot k-k+2}, \dots, \pi_{i\cdot k}\}
\text{where } {\mathbf F}^i = \{\pi_j | \left(i-1\right)\cdot k < j \leq i\cdot k \}.
\end{equation*}
\end{definition}

$\vec{\pi} = \langle\pi_1, \pi_2, \dots, \pi_\ell\rangle$ can be any permutation of the index set $S_\ell$.
In other words, ${\mathcal F}_k$ is an FOS consisting of $\ell/k$ disjoint masks, and every mask contains $k$ indexes.
Since the locations of variables should not affect the behavior of optimization,
we use the permutation notation to eliminate the dependency of variable locations.

\begin{definition}\label{def:catFOS}
The {\bf{concatenated homogeneous FOS}} can be defined below.
\begin{equation*}
{\mathcal F}_{k_1, \dots, k_a} = {\mathcal F}_{k_1} \parallel {\mathcal F}_{k_2} \parallel \dots \parallel {\mathcal F}_{k_a},
\end{equation*}
where $\parallel$ is the concatenation operator, which concatenates the elements in the sets while preserving the order.
\end{definition}

Take the case $\ell = 6$ as an example.
$\langle \{1, 5\}, \{2, 4\}, \{3, 6\}\rangle $, 
$\langle \{2, 4, 5\}, \{1, 3, 6\}\rangle $, and
$\langle \{1, 2, 3, 4, 5, 6\}\rangle $
are three valid homogeneous FOSs.
They can be denoted by ${\mathcal F}_2$, ${\mathcal F}_3$, and ${\mathcal F}_6$ respectively.
A valid concatenation of them is
${\mathcal F}_{3,6,2}=\langle \{2,4,5\},\{1,3,6\},\{1,2,3,4,5,6\},\{1,5\},\{2,4\},\{3,6\}\rangle $.

\subsection{Test Problems}
The analyses and experiments are based on three maximization problems.
%We consider optimization is successful if and only if it finds the global optimum.
By successful optimization we mean that the global optimum is found.
The first problem is onemax:
\begin{equation*}
%f_{onemax}\left({\vec{x}}\right)=\sum\limits_{i=1}^\ell x_i
f_{onemax}\left({\vec{x}}\right)=u\left({\vec x}\right),
\end{equation*}
where $u\left({\vec x}\right)$ is the number of $1$s in $\vec x$.
%Apparently the global optimum of onemax is all $1$s.
Onemax is trivial, but optimization performance on onemax
indicates how well an algorithm deals with fully separable problems.

The second problem is one instance of the Royal Road functions~\cite{1992_RR},
the structure of which can be characterized by an FOS $\mathcal F$:
\begin{equation}
\label{eq:RR}
%f_{royal}\left(\vec{x}\right) = \sum_{\forall {\mathbf F}^i \in {\mathcal F}} R\left(\vec{x}_{{\mathbf F}^i}\right),
f_{royal}\left(\vec{x}\right) = \sum_{\forall {\mathbf F}^i \in {\mathcal F}} R\left(\vec{x}_{{\mathbf F}^i}\right),
\end{equation}
where
\begin{equation*}
R\left(\vec{x}\right) = 
\left\{
\begin{array}{l l}
1 & \text{if }u\left({\vec x}\right)=|{\vec x}|, \\ %\text{if all bits in $\vec{x}$ are $1$s},\\
0 & \text{otherwise}.
\end{array}
\right.
\end{equation*}
%We use $\vec{x}_{\mathbf F}$ to denote a vector consisting of the variables from $\vec x$ indicated by a mask $\mathbf F$,
Note that $\vec{x}_{\mathbf F}$ is the part of $\vec x$ indicated by a mask $\mathbf F$,
and $|\vec x|$ denotes the numbers of elements in $\vec x$.
%and the function $u\left({\vec x}\right)$ means the number of $1$s in $\vec x$.
%$|\vec{x}|$ represents the number of variables in $\vec{x}$.
Motivation of using the Royal Road function is introduced in Section~\ref{sec:nfe}.

The third problem is the deceptive-trap problem~\cite{1992_trap}, which can also be characterized by an FOS $\mathcal F$:
\begin{equation*}
f_{trap}\left(\vec{x}\right) = \sum_{\forall {\mathbf F}^i \in {\mathcal F}} T\left(\vec{x}_{{\mathbf F}^i}\right),
\end{equation*}
where
\begin{equation*}
T(\vec{x}) = 
\left\{
\begin{array}{l l}
1 & \text{if $u(\vec{x}) = |\vec{x}|$}, \\
0.9 \cdot \frac{|\vec{x}|-1-u(\vec{x})}{|\vec{x}|-1} & \text{otherwise}.
\end{array}
\right.
\end{equation*}
%This is also a common benchmark problem.
%It is known for the deceptive nature that local hill climbing does not lead to the global optimum.
The deceptive-trap function is also commonly used for benchmark and is known for its deceptive nature.
Local hill climbing does not lead to the global optimum when solving trap problems.

Note that although these problems are all binary-encoded,
our results can be applied to $\chi$-ary problems.

\subsection{GOMEA}
%Recently the genepool OMEA (GOMEA) has attracted much attention~\cite{2011_LS_MB_OM,2012_Goldman,2013_Martins}.
%We choose GOMEA as our research subject because it
GOMEA~\cite{2011_OMEA} is one of the first proposed OMEAs and can be integrated with arbitrary FOS.
In this paper, we focus on the variation operator of GOMEA, which is named gene-pool optimal mixing (GOM).
The pseudocode of GOM can be expressed in Algorithm~\ref{alg:GOM}.
%To examine the performance of GOM, a simplified version of the canonical GOMEA is adopted in our experiment.
Utilizing this operator,
the pseudocode of GOMEA can be expressed in Algorithm~\ref{alg:OMEA}.
The population is denoted by $\mathcal P=\{P_1, \dots, P_n\}$, and
the offspring are denoted by $\{O_1, \dots, O_n\}$.
%When all chromosomes in the population evolve into the same instance, %have the same values of variables,
%the optimization halts.
Optimization halts if the population converges to one instance.
In this paper, all experiments are done with an implementation of GOMEA.
For the experiments that need to determine the minimal required population size, a bisection procedure~\cite{2001_bisection} is adopted.

\begin{algorithm}[h]
\SetKwInOut{Input}{Input}
\SetKwInOut{Output}{Output}
\DontPrintSemicolon
%\SetAlgoLined
\caption{Genepool Optimal Mixing}
\label{alg:GOM}
%\begin{algorithmic}[1]
\Input{$f$: fitness function, $R$: receiver,\\
${\mathcal P}$: population, ${\mathcal F}$: FOS}
\Output{$O$: offspring}
\BlankLine
$O \gets R$ //{\it Copy to the offspring} \;
$B \gets R$ //{\it Copy to a buffer} \;
\For{$i=1$ to $|{\mathcal F}|$}{
    $r \gets$ random number from $1$ to $|{\mathcal P}|$\;
    $D \gets P_r$ //{\it Select the donor} \;
    $B_{{\mathbf F}^i} \gets D_{{\mathbf F}^i}$ //{\it Get the fragment of the donor} \;
%    \Call{EvaluateFitness}{$B$}
    \If {$f\left(B\right) \geq f\left(O\right)$}{
	$O_{{\mathbf F}^i} \gets B_{{\mathbf F}^i}$ //{\it Adopt the change if improved}\; }
	\Else{
	$B_{{\mathbf F}^i} \gets O_{{\mathbf F}^i}$ //{\it Reset the buffer if not improved} \;
	}
%    \EndIf
%\EndFor
}
\Return $O$
%\end{algorithmic}
\end{algorithm}

\begin{algorithm}[h]
%\SetAlgoLined
\SetKwInOut{Input}{Input}
\SetKwInOut{Output}{Output}
\DontPrintSemicolon
%\caption{Genepool Optimal Mixing Evolutionary Algorithm}
\caption{GOMEA}
\label{alg:OMEA}
%\begin{algorithmic}[1]
\Input{$f$: fitness function, $n$: population size,\\
${\mathcal F}$: FOS}
\Output{optimization solution}
\BlankLine
%\For{$i=1$ to $n$}
%    $P_i \gets $\Call{RandomInitialize}{}
%    $P_i \gets $ randomly initialize
%    \Call{EvaluateFitness}{$P_i$}
%\EndFor
randomly generate $n$ instances for $\mathcal P$\;
\While{$\neg$\textsc{ShouldTerminate}$\left({\mathcal P}\right)$}{
    \For{$i=1$ to $n$}{
	$O_i \gets $\Call{GenepoolOptimalMixing}{$f, P_i, {\mathcal P}, {\mathcal F}$}\;}
    \For{$i=1$ to $n$}{
	$P_i \gets O_i$\;}
}
%\EndWhile
%\Return \Call{TheBestIn}{$\mathcal P$}
\Return the best instance in $\mathcal P$
%\end{algorithmic}
\end{algorithm}

%In the canonical GOMEA~\cite{2011_OMEA},
%%steepest ascent hill climbing is applied to the initial population,
%the linkage information is used to build the FOS.
%The canonical GOMEA~\cite{2011_OMEA} uses the linkage information to build the FOS.
In practice, linkage detection techniques are applied to generate FOS.
In our scenario, this step is omitted.
%We adopt perfect model instead of learned one to prevent unnecessary noise,
%Hill climbing is neglected, since it does not affect the mixing operator.
We manually determine the FOS before optimization to prevent unnecessary noise.
We also disable the function of mutation in all of our experiments,
which is the same as the scenario in the canonical GOMEA.


%\subsection{Previous Results of Population Sizing and Convergence Time}
\subsection{Population Sizing}

%For long the importance of building blocks~\cite{1975_Holland,1989_Goldberg} have been noticed. 
For genetic algorithms (GAs), the supply model was proposed based on the concept of 
%Successful design of GAs can be decomposed into several aspects~\cite{1992_Goldberg},
%and one of them is
ensuring an adequate supply of partial solutions.
GAs find the global optimum by mixing the segments of it,
which must exist in the initial population.
Consider a $\chi$-ary problem which can be decomposed into $m$ parts,
and each of them contains $k$ variables.
Since GAs are stochastic processes, we can never guarantee the existence of all partial solutions.
Instead, we consider a tolerable probability $\alpha$ of not having a partial solution in the initial population.
By approximating $\alpha$ by $1/m$, the reciprocal of number of subproblems, 
%Larger population implies higher probability of having all the required partial solutions,
Goldberg~{\it et~al}.\ derived a minimal required population size in this scenario~\cite{2001_Supply}:
%\begin{equation*}
%n=\chi^k\left(k\ln\chi-\ln \alpha\right).
%\end{equation*}
%To further simplify this equation, a reasonable assumption is that $\alpha$
%is inversely proportional to $m$.
%In other words, substitute $\alpha$ with $1/m$.
%Hence the required population size can be expressed as
\begin{equation}\label{eq:goldbergSupply}
n=\chi^k\left(k\ln\chi+\ln m\right).
\end{equation}

There are other population-sizing models for GAs.
Goldberg~{\it et~al}.\ proposed one by considering that
larger population leads to higher probability of making correct decisions~\cite{1992_Goldberg}.
Harik~{\it et~al}.\ proposed another model by making an analogy
between selection in GAs and the gambler's ruin problem~\cite{1999_Gamblers}.
These models give different estimations of the required population size.
In this paper, we find out that the supply model is a proper estimator for OMEAs.

%OM operators greatly reduce the required population size, and the above equation does not apply.
%Nevertheless, the concept of initial supply is similar in this paper.

%\subsection{Convergence Time}
%
%Convergence time means the number of generations needed
%until all instances in the population converge to the global optimum.
%The convergence time is highly related to the selection mechanism and the selection pressure.
%In the scenario of tournament selection with selection pressure $s=2$,
%Thierens and Goldberg have derived a simple estimation of convergence time~\cite{1993_Thierens}:
%\begin{equation}
%g_{conv}=\frac{\pi}{2}\sqrt{\pi \ell},
%\end{equation}
%where $\ell$ is the number of variables in a chromosome.
%
%In OMEA, no explicit selection are done, and this result does not apply.
%But analogy between OM operators and binary tournament selection can be made to retrieve similar result.


%% Chapter 3

%\newpage
\section{One-Layer Masks}
\label{sec:nonOverlappingMasks}

To begin our study, we first focus on the scenario so-called one-layer masks where FOS consists of disjoint masks.
Three assumptions are made in this section.
First, we only focus on problems which can be fully separated into subproblems.
Second, we assume that the problem structures are perfectly identified before optimization.
Discussion on not well-identified problems is in next section.
Third, we assume the global optimum is unique in every problem.
For problems with multiple global optimums, we expect smaller required population due to smaller need of initial supply.


%We also make two reasonable assumptions. The global optimum is unique, and the population is randomly initialized.
%In this scenario, we proposed the population-sizing model and the convergence-time model.
%Combination of these two models leads to the NFE model.

%Apparently the global optimum of $f_{onemax}$ is a chromosome with all bits being $1$s.
%All the variables are independent when evaluated,
%so this function can be further decomposed into smaller subfunctions.
%The variables can be partitioned into some disjoint sets.
%Assume there are $m$ such sets, $m\in{\mathbb N}$.
%The sets contains $k_1,k_2, \dots , k_m$ bits respectively,
%with $\sum\nolimits_{i=1}^m k_i = \ell$.
%A chromosome is the global optimum if and only if every variable is one,
%which implies every set is composed of all ones.
%Assume the FOS $\mathcal{F}$ consists of masks in correspondence to the sets,
%which means $\mathcal{F}=\{{\mathbf{F}}^1, \dots , {\mathbf{F}}^m\}$,
%with $|{\mathbf F}^i| = k_i$, $1\leq i\leq m$.
%%Consider the case that no mutation occurred.
%For this special case,
%the following results are derived.

\subsection{Population Sizing}

%Assume there is only one global optimum for the problem.
The required population size can be estimated from the viewpoint of initial supply.
Consider a set containing $k$ variables which are $\chi$-ary,
in correspondence to a mask with $k$ indexes.
We assume that the optimal state of the set is unique, which is part of the unique global optimum.
A set is considered as correct if it is of optimal state, and is incorrect otherwise.
%In other words, a correct set is a part of the global optimum.
%A set of variables is called an incorrect one if it is not correct.
The probability of the set being correct after
uniformly random initialization is $\chi^{-k}$.
Since the variables are exchanged using the corresponding mask,
they are always tied together during mixing.
%Therefore the mixing operation can be considered an exchange of existing variable pattern.
Therefore no new pattern is created after mixing.

Since the masks reflect the decomposed subproblems,
correct sets are never replaced,
and eventually dominate the population.
Hence the population converges to the optimal state for the mask
if and only if at least one correct set exists initially.
For a mask of size $k$,
the probability that at least one set of initial variables is correct among the population is
\begin{equation*}
p_{correct} = 1-\left(1-\chi^{-k}\right)^n.
\end{equation*}
%EA finds the global optimum if and only if all sets are in optimal state, 
EA finds the global optimum if and only if every subproblem contains correct sets initially,
the probability of which is
\begin{equation}\label{eq:pop_size_raw}
p_{success}=\prod\limits_{\forall {\mathbf{F}}^i \in {\mathcal{F}}} \left(1-\left(1-\chi^{-|{\mathbf{F}}^i|}\right)^n\right).
\end{equation}
If the tolerable failure rate of EA is $\alpha$,
solving $p_{success}=1-\alpha$ yields the required population size $n$.
%\begin{equation}
%1-\alpha=\prod\limits_{\forall {\mathbf{F}}^i \in {\mathcal{F}}} \left(1-\left(1-\chi^{-|{\mathbf{F}}^i|}\right)^n\right)
%\end{equation}

%To study further on the population size,
%consider the case the sizes of all disjoint sets are the same,
A special case of Equation~\ref{eq:pop_size_raw} is that all masks are equal-sized,
which means $|{\mathbf{F}}^i| = k$, $i={1,2, \dots, m}$.
Therefore the equation can be rewritten as
\begin{equation}\label{eq:suc_rate}
1-\alpha=p=\left(1-\left(1-\chi^{-k}\right)^n\right)^m.
\end{equation}
This result is verified empirically in Figure~\ref{fig:supply}.
We can see that the experiment values are accurately estimated.

Equation~\ref{eq:suc_rate} also leads to the required population size:
\begin{equation} \label{eq:pop_size}
n=\frac{\ln\left({1-\left(1-\alpha\right)^{\frac{1}{m}}}\right)}{\ln\left({1-\chi^{-k}}\right)}.
\end{equation}
%Now we consider its complexity with regarding to $m$.
%From this equation the complexity can be derived.
Note that $0<\frac{1}{m}\leq1$ in Equation~\ref{eq:pop_size}.
To derive the complexity of $n$,
we consider a convex function 
$f\left(x\right)=\left(1-\alpha\right)^x$,
where $x\in(0,1]$.
Two inequalities can be derived:
%\begin{equation}\label{eq:ineq1}
%f\left(x\right)\geq f\left(0\right) + x f'\left(0\right) = 1 - \ln\frac{1}{1-\alpha}\cdot x,
%\end{equation}
%and
%\begin{equation}\label{eq:ineq2}
%f\left(x\right)\leq\left(1-x\right)f\left(0\right)+x f\left(1\right) = 1-\alpha x.
%\end{equation}
%
\begin{equation*}
\left(1-x\right)f\left(0\right)+x f\left(1\right)\geq f\left(x\right)\geq f\left(0\right)+x f'\left(0\right)
\end{equation*}
%Thus we have the following inequality.
Let $x=\frac{1}{m}$ and take natural log, we have
%Combining these inequalities and the fact that the natural logarithm is strictly increasing yields
\begin{equation*}
\frac{\ln\frac{1}{1-\alpha}}{m}\leq 1-\left(1-\alpha\right)^\frac{1}{m} \leq \frac{\alpha}{m},
\end{equation*}
%\begin{equation*}
%\ln\left(\frac{ \ln\frac{1}{1-\alpha}}{m}\right) \geq\ln\left(1-\left(1-\alpha\right)^\frac{1}{m}\right) \geq \ln\left(\frac{\alpha}{m}\right),
%\end{equation*}
\begin{equation*}
\ln\left( c_1 \frac{1}{m}\right) \geq\ln\left(1-\left(1-\alpha\right)^\frac{1}{m}\right) \geq \ln\left(c_2 \frac{1}{m}\right),
\end{equation*}
where $c_1$ and $c_2$ are some constants.

%For other variables in Equation~\ref{eq:pop_size},
%$\alpha$ is the tolerable failure rate, and should not be changed.

%Under the assumption that the global optimum is unique, the results could be extended to any decomposable problems.
To compare the optimization cost of problems with different sizes, we assume that the problems have similar structures,
which implies they are encoded in the same way ($\chi$ is constant),
and the adopted masks are similar ($k$ is constant).
The tolerable failure rate $\alpha$ should not vary with the problem size.
By fixing $\alpha$, $\chi$, and $k$,
asymptotically tight bound for Equation~\ref{eq:pop_size} with regarding to $m$ can be derived:

\begin{equation*}
\Theta\left(n\right)=\Theta\left(\ln\left(1-\left(1-\alpha\right)^\frac{1}{m}\right)\right) = \Theta\left(\ln m\right).
\end{equation*}

The complexity matches the supply model in Equation~\ref{eq:goldbergSupply}, which also leads to $n=\Theta\left(\ln m\right)$.
In contrast,
models concerning decision making suggest population sizing from $\Theta\left(\sqrt{m}\right)$ to $\Theta\left(m\right)$~\cite{1992_Goldberg, 1999_Gamblers}.
%the decision-making model proposed by Goldberg~{\it et~al}.\ leads to $n=\Theta\left(m\right)$~\cite{1992_Goldberg}.
%Harik~{\it et~al}.\ proposed the gambler's ruin model with $n=\Theta\left(\sqrt{m}\right)$~\cite{1999_Gamblers}.
By comparing the complexities, we verified that the concept of supply suits well for OMEAs, while others do not.

%The figure indicates that the initial supply model is a good estimator of required population size.
%The results show that the successful rates is well estimated only depends on the supply model.
%Therefore the population size only depends on the initial supply instead of other aspects, such as decision-making.
%
%An experiment is conducted to verify our model.
%A 500-bit onemax problem is used, with an FOS consisting of 100 masks.
%Each of the masks contains 5 variables.
%For various population size, optimization is repeated 10000 times, and the number of successful runs is recorded.
%The result is shown in Figure~\ref{fig:supply},
%where the theoretical values are given by Equation~\ref{eq:suc_rate}.
%The experiment result indicates that
%the initial supply model along is a good estimator of the required population size for disjoint masks.

\begin{figure}%[H]
\centering
\epsfig{file=supply.eps, width=3in}
\caption{
Success rate for various population sizes with $\ell =500$ and $k=5$.
Experiments are repeated $10^4$ times.
The maximum absolute error is $1.03\%$.
}
\label{fig:supply}
\end{figure}

\subsection{Convergence Time}

%The study of convergence time involves two parts.
%The case of a single mask is first studied,
%the result of which then extends to multiple masks.
%Convergence time is the number of generations for OMEAs to reach the global optimum.
The study of convergence time, the number of generations for OMEAs to converge, involves two parts.
The case of a single mask is first studied,
and the result of which is then extended to two-layer masks.

For the one-mask case,
denote the mask size by $k$ and the generation number by $t$, starting with $t=0$.
%The mask size $k$ equals to $\ell$,
%which corresponds to all the variables.
%Note that $t\in{\mathbb N}_0$, the set of natural numbers including zero.
%$t$ is zero in the initial state.
%After each generation, $t$ is increased by 1.
$t$ increases 1 after each generation.
Define $p_t$ as the proportion of correct sets among all chromosomes.
%Assume that the global optimum is unique and the population is randomly initialized,
$p_0$ can be approximated by its expected value, $\chi^{-k}$.
%By modeling the growth of $p_t$ we can estimate the convergence time.
We estimate the convergence time by modeling the growth of $p_t$.
%The convergence time can be derived by 
%solving $p_{t_{conv}} = 1$.
%An approximation can be derived by solving the equation iteratively.

%In GOMEA, a variable set of a candidate is selected, and a set from another candidate is then selected,
%trying to donate the bit pattern to the previous one.
In GOMEA, two candidates are selected as the donor and receiver,
and the former tries to donate part of its bit pattern to the latter.
%Since there is only one set of variables,
The receiver takes the donation only if its fitness does not decrease,
so a correct set never accepts an incorrect one.
%On the other hand, an incorrect set might be overwritten by a correct one.
%When an incorrect set is taken as the receiver,
On the contrary, incorrect receiver becomes correct once the donor is correct.
%$P\left(\text{donor is correct}\right)=p_t$.
%On the other hand, the probability that a correct set overwrites an incorrect one is
%$P\left(\text{donor is correct}\right)\cdot P\left(\text{receiver is incorrect}\right) = p_t\cdot\left(1-p_t\right)$.
%Here we make the assumption that $p_t$ does not vary much during one generation.
Since the expected number of incorrect chromosomes in generation $t$ is $n\left(1-p_t\right)$,
%Since there are $n\left(1-p_t\right)$ incorrect chromosomes in generation $t$,
%Note that the change is not really made on the receiver, but its offspring instead.
%During the mixing stage, the variables in population remain constant while producing the offspring.
%Therefore $p_t$ remains constant during one generation.
%In a generation, each candidate in population is seleted as receiver exactly once.
%There are total $n$ chromosomes, and the expected number of incorrect sets is $n\cdot \left(1-p_t\right)$,
%so
the expected gain in the number of correct sets after one generation is
$n\left(1-p_t\right)\cdot Pr\left(\text{donor is correct}\right)
=n p_t\left(1-p_t\right)$.
Note that changes are made on the offspring instead of receiver,
so $p_t$ remains constant until the end of the current generation.
Thus we get the iterative equation:
\begin{equation}
\label{eq:p_ite}
n p_{t+1}-n p_t = n\cdot p_t\cdot\left(1-p_t\right).
\end{equation}
For $t\in{\mathbb N}_0 = {\mathbb N} \cup \{0\}$,
denote $1-p_t$ by $q_t$, and we have
\begin{equation*}
q_t-q_{t+1} = \left(1-q_t\right)q_t,
\end{equation*}
\begin{equation*}
%\Rightarrow
q_t^2 =  q_{t+1}.
\end{equation*}
Along with $q_0=1-p_0=1-\chi^{-k}$, we get
\begin{equation} \label{eq:p_t}
p_t = 1-q_t=1-\left(1-\chi^{-k}\right)^{2^t}, t\in{\mathbb N}_0.
\end{equation}

Although the solution to Equation~\ref{eq:p_ite} is derived,
no $t$ satisfies $p_t=1$,
because we approximate $p_t$ by its expected value.
Since $p_t$ represents the proportion of correct instances in the population with $n$ candidates,
the greatest value of $p_t$ less than $1$ is $1-1/n$. %$1-\frac{1}{n}$.
%This the limit of continuous estimation.
Since the expected time for $p_t$ to reach $1-1/n$ is less than the convergence time,
we derive a lower bound for the latter by solving $p_{t_{conv}} > 1-1/n$.
%Hence the estimation of convergence time is decomposed into two parts.
%First, we estimate the time $t_1$, the required time for $p_t$ to grow to $1-1/n$. %$1-\frac{1}{n}$.
%Second, we estimate the time $t_2$,
%which means how long it takes for the remaining incorrect chromosome to converge to the correct one.
%Hence the estimation of convergence time equals to $t_1+t_2$.

%Estimation of $t_1$ can be derived 
To estimate the lower bound $t_L$,
we approximate $p_t$ by a continuous function
%The definition of $p_t$ can be extended to
$p\left(t\right)$, which treats $t$ as a real number.
From Equation~\ref{eq:p_t}, we set
\begin{equation}
p\left(t\right) = 1 - q\left(t\right) = 1 - \left(1-\chi^{-k}\right)^{2^t}, t \geq 0.
\end{equation}
Note that $p\left(t\right) = p_t$ for $t\in{\mathbb N}_0$.
%and $p\left(t\right)$ is a monotonically increasing function.
%This equation can be considered an approximation of the sequence $p_t$.
Solving $p\left(t_{conv}\right)>1-\frac{1}{n}$ yields
%from the equation above, we get
\begin{equation} \label{eq:conv1}
t_{conv} > t_L = \log_2 \left(\frac{\ln\left( \frac{1}{n}\right)}{\ln\left(1-\chi^{-k}\right)}\right).
\end{equation}
%where $\lg$ is the binary logarithm.

%%Then we consider $t_2$.
%When only one incorrect set remains,
%we simply approximate the required time by one generation,
%since the population converges within one generation in most times.
%Therefore we have $t_2=1$.

%
%it becomes correct in the next generation if and only if the donor is another set rather than itself.
%%If the incorrect set acts as the donor itself, which is possible in GOMEA, it remains incorrect.
%Therefore after one generation, the status remains unchanged with a probability $1/n$.
%Estimating $t_2$ by its expected value yields
%%\begin{equation*}
%%t_2 = 1 + \frac{1}{n} \cdot t_2,
%%\end{equation*}
%\begin{equation} \label{eq:t2}
%t_2 = \frac{n}{n-1}.
%\end{equation}

%Combining $t_1$ and $t_2$ yields %from Equations~\ref{eq:t1} and \ref{eq:t2} yields the estimation of convergence time:
%\begin{equation}
%\label{eq:conv1}
%%t_{conv} = t_1 + t_2 = \lg\left(\frac{\ln \frac{1}{n}}{\ln\left(1-\chi^{-k}\right)}\right) + \frac{n}{n-1}.
%t_{conv} = t_1 + t_2 = \log_2 \left(\frac{\ln \frac{1}{n}}{\ln\left(1-\chi^{-k}\right)}\right) + 1.
%\end{equation}

%Next, consider the case that multiple disjoint masks are used when mixing the chromosomes.
The result can be further extended to multiple disjoint masks.
%Since we assume that the masks well reflects the BBs,
%In the onemax problem,
%variables are mutually independent,
Since the problem is separable,
convergence in one position does not interfere with that in another position.
Convergence time is thus dominated by the variable set with the longest convergence time.

Assume that the chromosome consists of $m$ sets of variables,
convergence time of which are
%and the convergence time of them are
$t_{conv,1}, \dots, t_{conv,m}$ respectively.
We have $t_{conv} = \max_i\left({t_{conv,i}}\right)$.
According to Equation~\ref{eq:conv1},
smaller initial proportion leads to longer convergence time.
%a negative relationship exists between convergence time and proportion of correct initial variables.
For the case that each set contains $k$ variables,
we can approximate the maximum of $t_{conv,i}$ by finding the convergence time of the set with
least initial correct instances.
Let $X_1, \dots, X_m$ be $m$ independent and identically distributed random variables following the binomial distribution $B\left(n,\chi^{-k}\right)$.\footnote
{
$B\left(n,p\right)$ denotes the distribution with the
pmf $Pr\left(x\right)=
\left\{
\begin{array}{l l}
\frac{n!}{\left(n-x\right)!x!} p^x \left(1-p\right)^{n-x} & x\in\{0,1,\dots,n\}, \\
0 & \text{otherwise}.
\end{array}
\right.$
}
Denote the first order statistic by $X_{\left(1\right)}=\min_i\left(X_i\right)$. Let $E[X_{\left(1\right)}] = x_{\left(1\right)}$.
Hence for multiple masks we have
\begin{equation}
\label{eq:conv}
%t_{conv} = t_1 + t_2 = \lg\left(\frac{\ln\left(\frac{1}{n}\right)}{\ln\left(1-\frac{x_{\left(1\right)}}{n}\right)}\right) + \frac{n}{n-1}.
t_{conv} > t_L = 
\log_2 \left(\frac{\ln \left(\frac{1}{n}\right)}{\ln\left(1-\frac{x_{\left(1\right)}}{n}\right)}\right).
\end{equation}

The lower bound we derived is verified with results on the onemax problem.
Figure~\ref{fig:conv1} shows the case of one mask, where the lower bound is from Equation~\ref{eq:conv1}.
%The case of one mask and multiple masks are shown in Figures~\ref{fig:conv1} and \ref{fig:conv2} respectively,
%where the theoretical values are from Equations~\ref{eq:conv1} and \ref{eq:conv} respectively.
%In the first experiment, only one mask of size $5$ is used for mixing.
%as the scenario in Equation~\ref{eq:conv1}.
%The problem size is 5 bits, implying the mask contains 5 variables.
%The result is shown in Figure~\ref{fig:conv1}, where the theoretical values are derived from Equation~\ref{eq:conv1}.
\begin{figure}%[H]
\centering
\epsfig{file=conv.eps, width=3in}
\caption{
Convergence time with one mask of size $5$.
%For population size ranging from $100$ to $25600$,
%the maximum relative error is $4.99\%$; the maximum absolute error is $0.42$ generations.
For population size ranging from $200$ to $204800$,
the maximum difference between experiment values and the lower bound is $0.81$ generations.
The difference decreases for large population.
}
\label{fig:conv1}
\end{figure}
The case of multiple disjoint masks is shown in Figures~\ref{fig:conv2} and \ref{fig:conv3},
where the lower bound is from Equation~\ref{eq:conv}.
%Next, the case of multiple disjoint masks is shown in Figure~\ref{fig:conv2}, where
%theoretical results are derived from Equation~\ref{eq:conv}.
%empirical and theoretical results are shown in Figure~\ref{fig:conv2}.
\begin{figure}%[H]
\centering
\epsfig{file=conv2.eps, width=3in}
\caption{
Convergence time for $n=10000$.
For problem size ranging from $25$ to $800$,
the maximum difference is $0.85$ generations.
}
\label{fig:conv2}
\end{figure}

\begin{figure}%[H]
\centering
\epsfig{file=conv3.eps, width=3in}
\caption{
Convergence time for $\ell=100$ with masks of size 5.
For population size ranging from $200$ to $102400$,
%the maximum relative error is $13.3\%$;
%the maximum absolute error is $1.37$ generations.
the maximum difference between experiment values and the lower bound is $1.37$ generations.
The difference decreases for large population.
}
\label{fig:conv3}
\end{figure}

%We infer that one major reason for the deviation in our results is that experiment samples are integers.
%The experiment values seems close to integers.
%This also explains why empirical results do not grow smoothly in Figure~\ref{fig:conv1}.

%The results strictly satisfy the proposed lower bound.
The proposed lower bound strictly bounds the empirical results.
For large $n$, $1-1/n$ approaches $1$.
Since the lower bound is derived by solving $p\left(t_L\right) = 1-1/n$,
the lower bound should be close to the convergence time for large population.
This can be verified in Figures~\ref{fig:conv1} and \ref{fig:conv3}.
Since an almost converged population is likely to converge in one more generation,
the lower bound is about one generation below experiment results.

%Unlike the case of a single mask, the convergence time may decrease as the population size increases
%when multiple masks are adopted.
%This is mainly because of the $\frac{x_{\left(1\right)}}{n}$ term in Equation~\ref{eq:conv}.
%Due to the Central Limit Theorem, the first order statistics of the proportion of correct instances increases when $n$ is large.
%This phenomenon is observed and verified in our experiments, and shown in Figure~\ref{fig:conv3}.
In the case of multiple masks,
we find out that the convergence time decreases while $n$ increases when $n$ is small.
%when multiple masks are adopted.
This is mainly because of the $\frac{x_{\left(1\right)}}{n}$ term in Equation~\ref{eq:conv}.
According to the central limit theorem,
the first order statistics of the proportion of correct instances increases and approaches $\chi^{-k}$ when $n$ grows.
When $n$ is large, the $\ln{\left(\frac{1}{n}\right)}$ term dominates the lower bound, making it increase when $n$ grows.
This phenomenon is verified in Figure~\ref{fig:conv3}.



\subsection{Number of Function Evaluations}\label{sec:nfe}

%There are two steps in OMEA which requires fitness function evaluations.
%Two steps in OMEA require fitness function evaluations.
%First, all chromosomes in the population are evaluated once.
%Next, the chromosomes keeps being evaluated until all of them converge to a solution.
%The NFE for the first step, denoted by $n_{fe,1}$, equals to $n$, the size of population.
%What needs further study is $n_{fe,2}$, the NFE of the second step.
%The total number of NFE can be estimated by
The function evaluations of OMEAs consist of two parts.
The first is those during the initialization, and the second is those during OM.
Therefore the total NFE is
\begin{equation}
n_{fe} = n + n_{fe,OM}.
\end{equation}
We then model $n_{fe,OM}$ for the case of a single size-$k$ mask.

%For the case of a single mask of size $k$, the following result is derived.
%We start our study on the case that $m=1$, which means there is only one mask.
%Denote the density of correct solutions in generation $t$ by $p_t$.
As before, the proportion of correct subsolutions in the population at generation $t$ is $p_t$.
%At generation $t$, denote the proportion of correct solutions among the population by $p_t$.
Under the assumption of unique global optimum,
$\chi^k-1$ subsolutions are not optimal.
Denote each of their densities in the population by $q^{(1)}_t, \dots, q^{(\chi^k-1)}_t$, respectively.
In this generation, a receiver only needs to be evaluated when it differs from the donor.
Therefore the probability that a receiver needs the evaluation is
\begin{equation}\label{eq:e_t}
e_t = 1-p_t^2-\left(q^{\left(1\right)}_t\right)^2-\left(q^{\left(2\right)}_t\right)^2-\dots-\left(q^{\left(\chi^k-1\right)}_t\right)^2.
\end{equation}
We already have the approximation of $p_t$ in Equation~\ref{eq:p_t}.
Since $q^{(1)}_t, \dots, q^{(\chi^k-1)}_t\in [0,1]$, we have
\begin{equation*}
\frac{\left(\sum\limits_{i=1}^{\chi^k-1} q_t^{\left(i\right)} \right)^2}{\chi^k-1}
    \leq \sum\limits_{i=1}^{\chi^k-1} \left(q_t^{(i)}\right)^2
    \leq \left(\sum\limits_{i=1}^{\chi^k-1} q_t^{(i)}\right)^2.
\end{equation*}
With the condition that
\begin{equation*}
\sum\limits_{i=1}^{\chi^k-1} q_t^{(i)} = 1-p_t,
\end{equation*}
Equation~\ref{eq:e_t} leads to
\begin{equation}\label{eq:UL}
1-p_t^2-\frac{\left(1-p_t\right)^2}{\chi^k-1}
\geq e_t
\geq 1-p_t^2-\left(1-p_t\right)^2.
\end{equation}

The required NFE can be approximated by
\begin{equation*}
n_{fe,OM} = \sum\limits_{t=0}^{\infty} n\cdot e_t  = n \sum\limits_{t=0}^{\infty} e_t.
\end{equation*}
%Recall Equation~\ref{eq:p_t}.
By expanding the summation, we can derive the upper and lower bounds of $n_{fe}$.
For the upper bound, there is no closed form.
Since 
$\sum\limits_{t=0}^{\infty}\left(1-p_t^2-\frac{\left(1-p_t\right)^2}{\chi^k-1}\right)$ is a function of $k$ if $p_t$ is approximated by Equation~\ref{eq:p_t},
we denote it by a function $U\left(k\right)$,
where $U\left(k\right)$ is a function from $\mathbb{N}$ to $\mathbb{R}$.
$\mathbb{N}$ is the natural number set, and $\mathbb{R}$ is the real number set.
Here the variable $\chi$ is considered constant.
So we have
\begin{equation*}
n_{fe,OM} \leq n\cdot U\left(k\right).
\end{equation*}

There is a closed form of the lower bound:
%We could prove that
\begin{equation*}
\begin{aligned}
\sum\limits_{t=0}^{\infty}\left(1-p_t^2-\left(1-p_t\right)^2\right)
=\sum\limits_{t=0}^{\infty} \left(2 q_t - 2 q_t^2\right) \\
=\sum\limits_{t=0}^{\infty} \left(2q_t - 2q_{t+1}\right)
=2q_0.
\end{aligned}
\end{equation*}
This yields
\begin{equation*}
n_{fe,OM} \geq n\cdot 2q_0 = n\cdot2\left(1-\chi^{-k}\right) = n\cdot L\left(k\right),
\end{equation*}
where $L\left(k\right) = 2\left(1-\chi^{-k}\right)$.
Hence the NFE can be bounded by
\begin{equation*}
n\cdot U\left(k\right) \geq n_{fe,OM} \geq n\cdot L\left(k\right).
\end{equation*}

The results can be extended to arbitrary number of masks, $m$.
Since all of our masks are disjoint, evaluations of them are independent to each others.
So the total NFE is the summation of all required NFEs for each mask.
Therefore we get
\begin{equation*}
n\cdot \sum_{\forall {\mathbf{F}}^i \in {\mathcal{F}}} U\left(|{\mathbf{F}}^i|\right)
\geq
n_{fe,OM}
\geq
n\cdot \sum_{\forall {\mathbf{F}}^i \in {\mathcal{F}}} L\left(|{\mathbf{F}}^i|\right),
\end{equation*}
or equivalently,
\begin{equation}
n+n\cdot \sum_{\forall {\mathbf{F}}^i \in {\mathcal{F}}} U\left(|{\mathbf{F}}^i|\right) 
\geq
n_{fe}
\geq
n+n\cdot \sum_{\forall {\mathbf{F}}^i \in {\mathcal{F}}} L\left(|{\mathbf{F}}^i|\right).
\end{equation}
For the special case that the sizes of all masks are equal, we have
\begin{equation}
\label{eq:nfe}
n\cdot\left(1+mU\left(k\right)\right)
\geq
n_{fe}
\geq
n\cdot\left(1+mL\left(k\right)\right),
\end{equation}
where $k$ is the size of every mask,
and $m$ is the number of masks.

The problem nature affects the required number of evaluations.
In Equation~\ref{eq:nfe}, the equality of upper bound holds when $q_t^{\left(1\right)} = q_t^{\left(2\right)} = \dots = q_t^{\left(\chi^k-1\right)}$.
Consider one of our test problems, the Royal Road function. % in Equation~\ref{eq:RR}.
By definition, all suboptimals in one subproblem (the $R\left(\vec x\right)$ in Equation~\ref{eq:RR}) contribute equally to the fitness.
%When optimizing this function, one cannot tell two sets apart by fitness if none of them is optimal.
Because all suboptimals are equally competitive during OM,
proportions of all suboptimal instances are roughly equal.
Since the equality of upper bound holds,
this problem costs more function evaluations than others do.

%The following experiments are designed to verify the results.
The results are verified in the following experiments.
%In the first scenario, the number of masks $m$ is fixed to $100$,
%and the population size is $10^4$.
Figure~\ref{fig:nfe_k} shows NFE with masks of various sizes.
Figure~\ref{fig:nfe_n} shows the result with various population sizes.
The theoretical values are from Equation~\ref{eq:nfe}.
%and the result is shown
%There are 100 masks in this scenario, and each of them contains 5 indices.

\begin{figure}
\centering
\epsfig{file=nfe_k.eps, width=3in}
\caption{
$n_{fe}$ with different sizes of masks.
Each FOS contains $100$ masks, and the population size is fixed at $10^4$.
Note that the Royal Road, the onemax, and the trap function are identical for $k=1$.
}
\label{fig:nfe_k}
\end{figure}

\begin{figure}
\centering
\epsfig{file=nfe_n.eps, width=3in}
\caption{$n_{fe}$ with different sizes of population.
$m=100$ and $k=5$ for the test problems.
}
\label{fig:nfe_n}
\end{figure}

In these two experiments, we have made some observations.
First, the lower bound is generally far from experiment results when there are multiple suboptimals,
since we derive the lower bound by ignoring the cost of comparison between suboptimals
in Equation~\ref{eq:UL}.
Second,
%while all the experiment values are bounded by the lower bound,
%values of the Royal Road function slightly exceed the upper bound.
NFE of the Royal Road function exceeds the upper bound by $2.17\%$ when $n=200$ and by $0.02\%$ when $n=12800$
in Figure~\ref{fig:nfe_n}.
%We infer that the reason is the assumption of sufficient population size.
We believe that the reason is as follows.
The estimation of $p_t$ deriving from the expected value %a continuous approximation to an iterative function,
while $p_t$ is discrete is the reason.
For small $n$, the approximation is inaccurate.
This is verified in Figure~\ref{fig:pt_n}, which indicates that
%For solving a 500-bit onemax problem with 5-bit masks,
%$p_t$ is recorded for various population sizes.
smaller population leads to slower convergence, %makes the population converge more slowly,
implying additional consumption in function evaluations.
This explains why NFE is slightly underestimated, especially when $n$ is small.
\begin{figure}%[H]
\centering
\epsfig{file=pt_n.eps, width=3in}
\caption{$p_t$ for various population sizes.
The problem is a $500$-bit onemax problem, with masks of size $5$.
We can see smaller $n$ leads to slower convergence.}
\label{fig:pt_n}
\end{figure}
%The derived upper and lower bounds are based on approximation,
%so they are not absolute tight bounds but estimators of NFE.
%Evolutionary algorithms are stochastic processes,
%so it is not easy nor realistic to derive absolute bounds for its performance.
%Even though the results are not absolute bounds,
%they are still good estimator of the required NFE.
Third, NFE of the Royal Road problem is always the greatest,
and NFE of the trap problem is always the least.
This verifies that the subproblem affects the NFE, yet the difference is subtle.
Last, the upper bound seems a proper estimator for the NFE for all test problems, %regardless of the problems, 
with relative error less than $5\%$.
We infer that the high mask-wise selection pressure of OM
quickly filters out the incorrect sets,
and therefore the convergence behavior is not much affected from the subproblem structures.

\section{Two-layer Masks}
\label{sec:twoLayerNonOverlappingMasks}

In this section, we extend our study to two-layer masks as the first step to multi-layer masks.
%which are often adopted in OMEAs.
Multi-layer masks are usually adopted since in practice most problem structures are unknown and not fully separable.
%In practice, multi-layer masks are often adopted in EAs.
%Empirical results indicates that GOMEA incorporates well with
%the linkage-tree structure~\cite{2011_OMEA}, which is a hierarchical multi-layer FOS.
%Following the results from previous section,
%we extend our study to the scenario of two-layer masks as the foundations of multi-layer case.
In this section, we focus on the onemax problem with FOS in the form ${\mathcal F}_{k,1}$ with $k>1$. % is adopted. 
Note that the FOS is defined in Definition~\ref{def:catFOS}.
%Here we focus on the onemax problem.

%\subsection{Population Size}

The population converges to the global optimum if and only if all the variables converge to $1$s.
%On the other hand, optimization fails if any variable does not converge to $1$.
For an index $i$, consider all the variables $\vec{x}_i$ in every chromosomes. %should converge to $1$ to achieve global optimum.
%Among the whole population,
There are $n$ variables with this index.
We use the notation $p$ to denote the proportion of $1$s in these $n$ variables.
%Consider the two layer of FOS applied here, namely ${\mathcal F}_k$ and ${\mathcal F}_1$.

Since masks in a homogeneous FOS do not share common variables,
exactly one mask in ${\mathcal F}_k$ and one mask in ${\mathcal F}_1$ contain $i$.
%the variable $\vec{x}_i$ belongs to exactly one mask in ${\mathcal F}_k$, and to another in ${\mathcal F}_1$.
When mixing with one-bit masks, $1$s never change to $0$s,
but this is not the case of $k$-bit masks.
Since the fitness is evaluated when all $k$ variables are exchanged,
the overall fitness never decreases, but some variables can be ruined.
For example, the pattern {\tt 101} overwrites {\tt 010} when a 3-bit mask applies to them, %they are mixed with a 3-bit mask,
since the fitness increases by 1.
However, the second variable in {\tt 010}, which is one, is overwritten by a zero.
For a certain index, if all $n$ variables with the index in the population become $0$s,
%If all $n$ variables are zeros,
%it will fail to find the global optimum.
the optimization fails.

This phenomenon can be further quantified from the viewpoint of probability.
%consider the initial state of population.
By the assumption of random initialization, $p=2^{-1}$ in binary-coded problems.
For a set of $k$ variables covered by a $k$-bit mask,
the number of $1$s in the set can be expressed as a random variable.
If the first variable is zero, the random variable $X_0$ follows the Binomial distribution $B(k-1,2^{-1})$.
Recall that $k$ is an integer greater than one.
If the first variable is one, the random variable is $X_1=1+X_0$.
Consider a variable ${\vec x}_i$ equals to $0$ in the donor and equals to $1$ in the receiver.
${\vec x}_i$ in the receiver becomes $0$ with probability $Pr\left(X_1\leq X_0\right)$,
which we called the reverse-growth probability, $p_{rg}$.
%When optimizing the onemax problem, appearance of $0$s is considered harmful.
%Therefore larger $p_{rg}$ makes optimization fail with higher probability.
%The probability that $X_1 < X_0$ affects the probability that a correct variable is lost.
The probabilities are calculated using the Binomial distribution and shown in Table~\ref{tbl:x1x0}.
%We can see that higher $k$ leads to higher value of $P(X_1<X_0)$, which means higher probability of loss of correct copies.
We can see larger $k$ leads to larger $p_{rg}$. %, meaning higher probability of optimization failure.

\begin{table}
\centering
\begin{tabular}{|c|c|c|c|c|} \hline
$k$ & $Pr(X_1>X_0)$ & $Pr(X_1=X_0)$ & $Pr(X_1<X_0)$ & $p_{rg}$\\ \hline
2 & 75\% & 25\% & 0\%   & 25\%\\
3 & 69\% & 25\% & 6\%   & 31\%\\
4 & 66\% & 23\% & 11\%  & 34\%\\
5 & 64\% & 22\% & 14\%  & 36\%\\ \hline
\end{tabular}
\caption{Comparison of two random variables $X_1$ and $X_0$.
Larger $k$ yields larger $p_{rg}$, which is the probability that $X_1 \leq X_0$.
%Higher $k$ leads to higher probability of making wrong decision during mixing.
}
\label{tbl:x1x0}
\end{table}

%The probability that an undesired copy is made affects the required population size.
The required population size varies when adopting various FOSs.
The concept of cross competition explains this phenomenon~\cite{1992_CC}.
Cross competition is first introduced to derive an upper bound for the selection pressure in GA.
%and cross-competitive failure is likely to occur when the selection pressure is high:
%The upper bound is derived to prevent cross-competitive failure:
To prevent cross-competitive failure, the selection pressure must satisfy
\begin{equation*}
s<n\frac{\ln\left(1-p_0\right)}{\ln \alpha},
\end{equation*}
where $s$ is the selection pressure, $p_0$ is the probability that a correct bit is preserved into the next generation,
and $\alpha$ is a threshold, which means optimization is likely to fail when $\alpha\ell$ variables are incorrectly converged.
%the proportion of lost bits exceeds $\alpha$.
After some modification, we obtain
\begin{equation}\label{eq:CC}
n>\frac{s\cdot \ln \alpha}{\ln\left(1-p_0\right)}.
\end{equation}

Although in the scenario of GOMEA, the situation is different, but the similar concept applies.
Larger $k$ leads to larger $p_{rg}$,
which then leads to lower probability that a correct bit is preserved.
In other words, larger $k$ yields smaller $p_0$.
%to smaller $p_0$, since larger $k$ is more likely to make correct variables be overwritten.
By assuming that $s$ and $\alpha$ do not vary much with $k$,
the RHS of Equation~\ref{eq:CC} increases when $k$ increases.
This means that the lower bound of population size increases.
In other words, the required population increases if $k$ increases when adopting ${\mathcal F}_{k,1}$.

The following experiments are conducted to verify the above hypothesis.
For various lengths of problems, different FOSs are adopted to solve the onemax problem.
These FOSs are ${\mathcal F}_{5,1}$, ${\mathcal F}_{4,1}$, ${\mathcal F}_{3,1}$, ${\mathcal F}_{2,1}$, and ${\mathcal F}_1$.
The result is shown in Figure~\ref{fig:pop_k1},
indicating that optimization using ${\mathcal F}_{k,1}$ with larger $k$ requires larger population.
Furthermore,
a proportional relationship exists between the required population size for ${\mathcal F}_{k,1}$ and ${\mathcal F}_1$,
while the latter can be approximated by Equation~\ref{eq:pop_size}.
%we find out that for ${\mathcal F}_{k,1}$, the required population is proportional to that of ${\mathcal F}_1$,
From this observation,
the theoretical values of required population for ${\mathcal F}_{k,1}$ are obtained by fitting the experiment values to a constant multiple of that of ${\mathcal F}_1$.
According to the results, masks which cover multiple separable subproblems are not beneficial from the viewpoint of population size.

\begin{figure}%[H]
\centering
\epsfig{file=pop_k1.eps, width=3in}
\caption{Required population size for various FOS.
Theoretical values of ${\mathcal F}_{k,1}$
are estimated by curve fitting as multiple of that of ${\mathcal F}_1$.
The maximum relative error among all estimations is $2.0\%$.
}
\label{fig:pop_k1}
\end{figure}

%\subsection{Convergence Time}

%\subsection{Number of Function Evaluations}

%\section{Experiments}
%\section{Discussions}

\section{Conclusion}
\label{sec:conclusion}

In this paper, we derived the convergence models of OMEAs
for one-layer and two-layers masks.
For problems with separable structures,
behaviors of GOMEA with masks corresponding to the structures were analyzed.
The required population size was accurately estimated from the viewpoint of initial supply.
Analyzing the growth of sub-solution led to the convergence-time model.
The NFE was then estimated.
These three models were verified empirically,
and the values of relative error among all experiments were less than $5\%$.
For multi-layer disjoint masks, a special case with two-layer masks was studied.
We found that the concept of cross competition explains the growth in the required population sizes.

%On the topic of optimal mixing, there is still room for future works.
%Our results can be extended to GOMEA adopting more general FOSs,
%and the thoughts and methods in this paper may be further applied to other scenarios.
%Comparison of models of different algorithms is also worth further studying.
%We believe our work together with further studies may lead to more competent and reliable EAs.
%Possible future works includes the following.
As for future work, we would like to extend our results to multi-layer masks by quantifying the effect of cross competition and by investigating the growth of subsolutions iteratively.
Hopefully, we would be able to analyze the behavior of OMEAs with full linkage tree,
which is a special case of multi-layer FOS.
Furthermore, since this paper is limited by the assumptions we made, we would like to study the case with weaker assumptions.

%For two-layer masks,
%quantitative population models may be further derived, as well as the convergence-time models and NFE models.
%For multiple-layer masks,
%quantifying the effect of cross competition may lead to population-sizing models.
%By inspecting the mixing operator utilizing masks from different layer,
%the convergence behavior can be modeled as a set of iterative equations,
%leading to convergence-time models.
%Since a special case of multiple-layer FOS is the hierarchical linkage sets,
%models of the LTGA family may be derived.
%Furthermore, by considering overlapping masks as multiple disjoint-mask layers,
%one can derived models handling overlapping masks.
%Possible approaches may be further analyzing the effect of cross competition.

The major contributions of this paper reside in the derivations of
the population-sizing, the convergence-time,
and the NFE models for OMEAs with one-layer and two-layer masks.
Our models are empirically verified and the relative errors of our estimators are small.
%In addition, according to our models, the following important findings can be conducted.
In addition, our models lead to the following insightful findings.
First, for the case of one-layer masks, the required population size is decided by initial supply rather than decision making.
This explains why OMEAs generally require relatively small populations compared to EDAs.
Second, NFE for the test problems is very close to the proposed upper bound,
by which we infer that the mask-wise selection pressure of OM quickly filter out non-optimum subsolutions, making the subproblem composition insignificant.
Third, for two-layer masks,
the required population size is proportional to that of the one-layer masks,
and the ratio is positively related to the reverse-growth probability.
%First, for the case of one-layer masks,
%the supply model is an accurate estimator of the required population size,
%and the convergence time is accurately modeled by estimating the growth of subsolutions.
%This explains why OMEAs generally require relatively small populations compared to EDAs.
%Second,
%NFE for the test problems is very close to the proposed upper bound,
%by which we infer that the mask-wise selection pressure of OM
%quickly filter out non-optimum subsolutions,
%making the subproblem structure insignificant.
%%the problem characteristics affects the required NFE,
%%which can be bounded by our estimators.
%%the derived estimators successfully bound the required NFE which is affected by problem characteristics.
%Third, for two-layer masks, the required population size is proportional
%to results from the one-layer-mask case, and the ratio is positively related to
%the reverse-growth probability.
%%the probability of the failure of convergence due to cross competition.
%%the probability of cross competition failure.


%By deriving the performance models of GOMEA,
%users may gain knowledge to choose the proper optimization algorithms.
%The best algorithm may not exist,
%but different algorithms should be adopted in different situations.
%Knowing the cost models of available algorithms prevents users from choosing blindly.
%Furthermore, the essence of optimal mixing is analyzed in the paper.
%Further study and improvement could be done with knowledge of this operator.
%Third, similar concepts in the related works can apply in our work.
%The successful trial should 

\section{ACKNOWLEDGMENTS}

The authors would like to thank the support by Ministry of Science and Technology in Taiwan under Grant No.\ MOST 103-2221-E-002-177-MY2-1.

\bibliographystyle{abbrv}
%\bibliography{paper}

\begin{thebibliography}{10}

\bibitem{2012_LN_OM_FI}
P.~A. Bosman and D.~Thierens.
\newblock Linkage neighbors, optimal mixing and forced improvements in genetic
  algorithms.
\newblock {\em Proceedings of the Genetic and Evolutionary Computation
  Conference $($GECCO-2012\,$)$}, pages 585--592, 2012.

\bibitem{2013_Filtering}
P.~A. Bosman and D.~Thierens.
\newblock More concise and robust linkage learning by filtering and combining
  linkage hierarchies.
\newblock {\em Proceedings of the Genetic and Evolutionary Computation
  Conference $($GECCO-2013\,$)$}, pages 359--366, 2013.

\bibitem{1992_trap}
K.~Deb and D.~E. Goldberg.
\newblock Analyzing deception in trap functions.
\newblock {\em Proceedings of the Second Workshop on Foundations of Genetic
  Algorithms $($FOGA-1992\,$)$}, 2:98--108, 1992.

\bibitem{1992_Goldberg}
D.~E. Goldberg, K.~Deb, and J.~H. Clark.
\newblock Genetic algorithms, noise, and the sizing of populations.
\newblock {\em Complex Systems}, 6:333--362, 1991.

\bibitem{1992_CC}
D.~E. Goldberg, K.~Deb, and D.~Thierens.
\newblock Toward a better understanding of mixing in genetic algorithms.
\newblock {\em Urbana}, 51:61801, 1992.

\bibitem{2001_Supply}
D.~E. Goldberg, K.~Sastry, and T.~Latoza.
\newblock On the supply of building blocks.
\newblock {\em Proceedings of the Genetic and Evolutionary Computation
  Conference $($GECCO-2001\,$)$}, pages 336--342, 2001.

\vfill\eject

\bibitem{2012_Goldman}
B.~W. Goldman and D.~R. Tauritz.
\newblock Linkage tree genetic algorithms: variants and analysis.
\newblock {\em Proceedings of the Genetic and Evolutionary Computation
  Conference $($GECCO-2012\,$)$}, pages 625--632, 2012.

\bibitem{1999_Gamblers}
G.~Harik, E.~Cant{\'u}-Paz, D.~E. Goldberg, and B.~L. Miller.
\newblock The gambler's ruin problem, genetic algorithms, and the sizing of
  populations.
\newblock {\em Evolutionary Computation}, 7(3):231--253, 1999.

\bibitem{1992_RR}
M.~Mitchell, S.~Forrest, and J.~H. Holland.
\newblock The royal road for genetic algorithms: Fitness landscapes and {GA}
  performance.
\newblock {\em Proceedings of the First European Conference on Artificial
  Life}, pages 245--254, 1992.

\bibitem{1993_BGA}
H.~M{\"u}hlenbein and D.~Schlierkamp-Voosen.
\newblock Predictive models for the breeder genetic algorithm: {I}.
  {C}ontinuous parameter optimization.
\newblock {\em Evolutionary Computation}, 1(1):25--49, 1993.

\bibitem{2002_Pelikan}
M.~Pelikan, K.~Sastry, and D.~E. Goldberg.
\newblock Scalability of the {B}ayesian optimization algorithm.
\newblock {\em International Journal of Approximate Reasoning}, 31(3):221 --
  258, 2002.

\bibitem{2001_bisection}
K.~Sastry.
\newblock {\em Evaluation-relaxation schemes for genetic and evolutionary
  algorithms}.
\newblock PhD thesis, University of Illinois at Urbana-Champaign, 2001.

\bibitem{2010_LTGA}
D.~Thierens.
\newblock Linkage tree genetic algorithm: First results.
\newblock {\em Proceedings of the 12th Annual Conference Companion on Genetic
  and Evolutionary Computation}, pages 1953--1958, 2010.

\bibitem{2011_OMEA}
D.~Thierens and P.~A. Bosman.
\newblock Optimal mixing evolutionary algorithms.
\newblock {\em Proceedings of the Genetic and Evolutionary Computation
  Conference $($GECCO-2011\,$)$}, pages 617--624, 2011.

\bibitem{2013_Hier}
D.~Thierens and P.~A. Bosman.
\newblock Hierarchical problem solving with the linkage tree genetic algorithm.
\newblock {\em Proceedings of the Genetic and Evolutionary Computation
  Conference $($GECCO-2013\,$)$}, pages 877--884, 2013.

\bibitem{1993_Thierens}
D.~Thierens and D.~E. Goldberg.
\newblock Mixing in genetic algorithms.
\newblock {\em Proceedings of the Fifth International Conference on Genetic
  Algorithms $($ICGA-1993\,$)$}, pages 38--47, 1993.

\bibitem{2014_CP}
S.-M. Wang, Y.-F. Tung, and T.-L. Yu.
\newblock Investigation on efficiency of optimal mixing on various linkage
  sets.
\newblock {\em Proceedings of the 2014 IEEE International Conference on
  Evolutionary Computation}, pages 2475--2482, 2014.

\bibitem{2007_Yu}
T.-L. Yu, K.~Sastry, D.~E. Goldberg, and M.~Pelikan.
\newblock Population sizing for entropy-based model building in discrete
  estimation of distribution algorithms.
\newblock {\em Proceedings of the Genetic and Evolutionary Computation
  Conference $($GECCO-2007\,$)$}, pages 601--608, 2007.

\end{thebibliography}

\end{document}
